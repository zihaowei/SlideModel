% !TEX program = xelatex
% !Mode::"TeX:UTF-8"
% !Mode:: "TeX:UTF-8"
\documentclass[xcolor=svgnames,serif,table,10pt,aspectratio=43]{beamer} %设置为 Beamer 文档类型,设置字体为 10pt,长宽比为16:9,数学字体为 serif 风格
\mode<presentation>{
% Setup appearance:
\useoutertheme{infolines}
\usetheme{CambridgeUS} %主题
\usecolortheme{dolphin} %主题颜色
\setbeamercovered{transparent}
\setbeamertemplate{caption}[numbered]
\setbeamertemplate{navigation symbols}{}
\setbeamertemplate{blocks}[rounded][shadow=true]
\setbeamertemplate{enumerate items}[circle]

% 修改样式
\setbeamercolor{box}{bg=black!20!orange,fg=white}
\setbeamercolor{block title}{use=sidebar,fg=sidebar.fg!10!white,bg=orange!70!black}
\setbeamercolor{block title example}{use=sidebar,fg=sidebar.fg!10!white,bg=black!60!green}
\setbeamercolor{block title alerted}{use=sidebar,fg=sidebar.fg!10!white,bg=black!50!red}

% \setbeamertemplate{headline}
% {%
    % \begin{beamercolorbox}[shadow=true]{section in head/foot}
    % \vskip2pt\insertnavigation{\paperwidth}\vskip2pt
    % \end{beamercolorbox}%
% }
}
\usepackage{url}
\usepackage{animate}
\usepackage[english]{babel}
\usepackage{times}
\usepackage[T1]{fontenc}
\usepackage{multirow,multicol,longtable}
\usepackage{graphics}
\usepackage{xcolor}
\usepackage[no-math]{fontspec}%-------------------------------------------------- 提供字体选择命令
\usepackage{xunicode}%----------------------------------------------------------- 提供Unicode字符宏
\usepackage{xltxtra}%------------------------------------------------------------ 提供了针对XeTeX的改进并且加入了XeTeX的LOGO
\usepackage[BoldFont,SlantFont,CJKchecksingle]{xeCJK}%--------------------------- 使用xeCJK宏包
%================================== 设置中文字体 ================================%
\setCJKmainfont{SimHei}%------------------------------------------------设置正文字体,Windows下采用黑体 SimHei,Mac下采用华文黑体 STHeiti
\setCJKmonofont{Adobe Song Std}%-------------------------------------------------设置等距字体
\setCJKsansfont{Adobe Kaiti Std}%------------------------------------------------设置无衬线字体
% \setCJKfamilyfont{zxzt}{FZShouJinShu-S10S}
% \setCJKfamilyfont{FZDH}{FZDaHei-B02S}
%================================== 设置中文字体 ================================%

%================================== 设置英文字体 ================================%
\setmainfont[Mapping=tex-text]{Times New Roman}%--------------------------------英文衬线字体
\setsansfont[Mapping=tex-text]{Arial}%------------------------------------英文无衬线字体
\setmonofont[Mapping=tex-text]{Courier New}%-------------------------------------英文等宽字体
\newfontfamily\Arial{Arial}
%================================== 设置英文字体 ================================%

%================================== 设置数学字体 ================================%
%\setmathsfont(Digits,Latin,Greek)[Numbers={Lining,Proportional}]{Minion Pro}
%================================== 设置数学字体 ================================%
\punctstyle{kaiming}%------------------------------------------------------------ 开明式标点格式
\usepackage{graphicx}
\usepackage{tikz}
\usetikzlibrary{positioning,backgrounds}
\usetikzlibrary{fadings}
\usetikzlibrary{patterns}
\usetikzlibrary{calc}
\usetikzlibrary{shadings}
\pgfdeclarelayer{background}
\pgfdeclarelayer{foreground}
\pgfsetlayers{background,main,foreground}
\usepackage{xifthen}
\usepackage{colortbl,dcolumn}
\usepackage{enumerate}
\usepackage{pifont}
\usepackage{tabularx}
\usepackage{booktabs}
\usepackage{hyperref}
%=================================== 数学符号 =================================%
\newcommand{\rtn}{\mathrm{\mathbf{R}}}
\newcommand{\N}{\mathrm{\mathbf{N}}}
\newcommand{\As}{\mathrm{a.s.}}
\newcommand{\Ae}{\mathrm{a.e.}}
\newcommand*{\PR}{\mathrm{\mathbf{P}}}
\newcommand*{\EX}{\mathrm{\mathbf{E}}}
\newcommand{\EXlr}[1]{\mathrm{\mathbf{E}}\left[#1\right]}
\newcommand*{\dif}{\,\mathrm{d}}
\newcommand*{\F}{\mathcal{F}}
\newcommand*{\h}{\mathcal{H}}
\newcommand*{\vp}{\varepsilon}
\newcommand*{\prs}{\dif\PR-\As}
\newcommand*{\dte}{\dif t-\Ae}
\newcommand*{\pts}{\dif\PR\times\dif t-\Ae}
\newcommand{\Ito}{It\^{o}}
\newcommand{\tT}[1][0]{[#1,T]}
\newcommand{\intT}[2][T]{\int^{#1}_{#2}}
\newcommand{\intTe}[1][t]{\intT[t+\varepsilon]{#1}}
\newcommand{\s}{\mathcal{S}}
\newcommand{\me}{\mathrm{e}}
\newcommand{\one}[1]{{\bf 1}_{#1}}
\renewcommand{\M}{{\rm M}}
\newcommand{\Me}[1][t]{M^{\varepsilon}_{#1}}
\newcommand{\Ne}[1][t]{N^{\varepsilon}_{#1}}
\newcommand{\Pe}[1][t]{P^{\varepsilon}_{#1}}
\DeclareMathOperator*{\sgn}{sgn}
% =================================== 数学符号 =================================%

% 定义罗马数字
\makeatletter
\newcommand{\rmnum}[1]{\romannumeral #1}
\newcommand{\Rmnum}[1]{\expandafter\@slowromancap\romannumeral #1@}
\makeatother

% 定义破折号
\newcommand{\pozhehao}{\kern0.3ex\rule[0.8ex]{2em}{0.1ex}\kern0.3ex}
% 中文日期
\def\CJK@today{\the\year 年 \the\month 月 \the\day 日}
\newcommand\zhtoday{\CJK@today}

% 中文图表
\renewcommand\figurename{图}
\renewcommand\tablename{表}

\graphicspath{{./}}

%题目,作者,学校,日期
\title{分组密码算法S盒的紧凑实现}
\subtitle{\textbf{Compact Implementation of Block Cipher S-boxes}}

\author[魏子豪]{答辩人: 魏子豪 \newline \newline 指导老师: 胡磊研究员}

\institute[IIE,UCAS]{
    % \includegraphics[height=1cm]{logo.jpg}
    \begin{figure}[h]
        \begin{minipage}[b]{0.35\linewidth}
        \centering
        \includegraphics[width=0.7\textwidth]{Figure/iie.jpg}
        \end{minipage}   
        \begin{minipage}[b]{0.35\textwidth}
        \centering
        \includegraphics[width=1\textwidth]{Figure/logo_ucas.png}
        \end{minipage}
    \end{figure}
}

\date{\zhtoday}

\setlength{\baselineskip}{22pt}
\renewcommand{\baselinestretch}{1.4}

% The main document

\begin{document}

\setlength{\abovedisplayskip}{1ex}%------------------------------------------ 公式前的距离
\setlength{\belowdisplayskip}{1ex}%------------------------------------------ 公式后的距离

\begin{frame}
	\titlepage
\end{frame}


\begin{frame}
	\frametitle{报告内容}
	\tableofcontents[hideallsubsections]
\end{frame}


% 每节开始插入目录页面
\AtBeginSection[]{
	\begin{frame}
		\frametitle{报告内容}
		\tableofcontents[currentsection, hideallsubsections]
	\end{frame}
	\addtocounter{framenumber}{-1}  %目录页不计算页码
}


\AtBeginSubsection[] {
	\begin{frame}
		\frametitle{报告内容}
		\tableofcontents[sectionstyle=show/shaded, subsectionstyle=show/shaded/hide]
	\end{frame}
	\addtocounter{framenumber}{-1}  %目录页不计算页码
}

\section{第一部分}
\begin{frame}
	一些内容
\end{frame}
\begin{frame}
	一些内容
\end{frame}
\section{第二部分}
\begin{frame}
	一些内容
\end{frame}
\begin{frame}
	一些内容
\end{frame}
\section{第三部分}
\begin{frame}
	一些内容
\end{frame}
\begin{frame}
	一些内容
\end{frame}




\begin{frame}[plain]{}
	\begin{center}
		\begin{tikzpicture}
			\node[above,xscale=1.2,yscale=1.2]{\Huge 欢迎批评指正!};
		\end{tikzpicture}
	\end{center}
\end{frame}

\end{document}





%%%%下面的内容不参与文档的编译。使用者在想用某个东西时直接可通过查阅,并复制黏贴和修改使用。

\iffalse  %注释开始

	%垂直居中
	\begin{frame}
		\begin{center}
			需要居中的内容!
		\end{center}
	\end{frame}
	或者
	\begin{frame}
		\centering
		一些内容
	\end{frame}

	%幻灯片标题的使用
	\begin{frame}
		\frametitle{第一部分第一张幻灯}
		一些内容
	\end{frame}

	%项目编号的使用
	\begin{frame}
		\frametitle{条目}
		\begin{itemize}
			\item 项目1
			\item 项目2
			\item 项目3
			\item 项目4
			      \begin{itemize}
				      \item 二级项目1
				      \item 二级项目2
			      \end{itemize}
		\end{itemize}
	\end{frame}

	%表格的使用
	\begin{frame}
		\frametitle{表格}
		\begin{table}[htbp!]
			\centering
			\caption{主流机器学习框架}
			\begin{tabular}{c|c|c|c|c}
				\toprule[1pt]
				机器学习库 & 机构     & 支持语言    & 平台   & Tensor \\
				\toprule[1pt]
				TensorFlow & Google   & C++,Python & 跨平台 & Good   \\
				\hline
				Pytorch    & Facebook & Python      & 跨平台 & Good   \\
				\bottomrule[1pt]
			\end{tabular}
		\end{table}
	\end{frame}

	%区块的使用
	\begin{frame}
		\frametitle{分析}
		\begin{block}{XXX 算法}
			\begin{itemize}
				\item 步骤1
				\item 步骤2
				\item 步骤3
			\end{itemize}
		\end{block}
	\end{frame}

	%使用区块来强调内容
	\begin{frame}
		\frametitle{强调}
		\begin{itemize}
			\item 这是内容
		\end{itemize}
		\only<1>\begin{block}{}
			这里蹦出来一个强调!
		\end{block}
	\end{frame}

	%section中目录的使用
	\begin{frame}
		\frametitle{技术影响力}
		\tableofcontents[currentsection,hideallsubsections]
	\end{frame}

	%插入图片
	\begin{frame}
		\begin{figure}[!h]
			\centering
			% Requires \usepackage{graphicx}
			\includegraphics[width=2cm]{pics/logo.jpg}\\
			\caption{logo图片样例}\label{pic6}
		\end{figure}
	\end{frame}

	%分栏实现图文混排
	\begin{frame}
		分栏前面的一些内容!!
		\begin{columns}%0.6 0.4表示相对比例
			\column{0.6\textwidth}%<1->
			分栏的左侧,文字叙述。
			\column{0.4\textwidth}%<1->
			分栏的右侧插入了图片。
			\begin{figure}[!h]
				\centering
				% Requires \usepackage{graphicx}
				\includegraphics[width=4cm]{pics/logo.jpg}\\
				\caption{logo图片样例}\label{pic6}
			\end{figure}
		\end{columns}
		分栏后面的一些内容!!
	\end{frame}

\fi   %注释结束
